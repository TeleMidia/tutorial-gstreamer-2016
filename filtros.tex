\section{Filtros}
\label{sec:filtros}
Filtros são elementos que manipulam amostras de áudio ou vídeos, sendo 
geralmente utilizados para criar efeitos audiovisuais em um pipeline. Eles
sempre possuem pelo menos duas \en{pads}, sendo uma para receber os dados
sobre os quais operam (\en{sink pad}) e outra para escrever os dados 
resultantes do processamento (\en{source pad})\footnote{Alguns filtros possuem mais de uma \en{pad} de entrada e/ou saída.}. A distribuição oficial do
GStreamer possui vários filtros disponíveis, dentre os quais: \en{volume}
(altera o volume de um fluxo de áudio), \en{videoscale} (altera a resolução de
um fluxo de vídeo), \en{adder} (recebe vários fluxos de áudio e gera um único 
fluxo contendo os sinais de entrada somados), \en{audiokaraoke} (remove a voz
de um fluxo de áudio), etc.

Para demonstrar como podemos adicionar filtros em um \en{pipeline}, vamos 
modificar o programa apresentado na Listagem~\ref{lst:mp3} e alterá-lo para 
criar um programa que reproduz um áudio MP3 com a metade do volume original.
O código do reprodutor de MP3 modificado é apresentado na 
Listagem~\ref{lst:mp3-volume}.

\lstinputlisting[
style=display,
mathescape=no,
caption={Tocando um arquivo de áudio MP3 com a metade do volume original.},
label={lst:mp3-volume},
]{src/mp3-volume.c}

Note que este programa é bastante semelhante ao programa da
Listagem~\ref{lst:mp3}. A diferença entre ambos está no fato do programa da
Listagem~\ref{lst:mp3-volume} adicionar o elemento \en{volume} ao 
``pipeline''. Este elemento, instanciado na linha~21, funciona como um
filtro e está posicionado entre o decodificador (``mad'') e o \en{sink}
(``alsasink''). Sendo assim, o filtro \en{volume} processa o fluxo de áudio
gerado pelo decodificador e gera um fluxo de áudio modificado para o 
\en{sink}. Na linha~28, a chamada \C{g_object_set} atribui o valor \C{0.5} 
à propriedade ``volume'' do filtro. Essa propriedade controla o nível do 
volume do fluxo de áudio produzido por este filtro, sendo o valor \C{0.0} 
correspondendo ao menor nível (mudo) e \C{1.0} correspondendo ao nível 
máximo  (volume original). Como o valor atribuído é \C{0.5}, o filtro 
processa o fluxo de entrada e gera um fluxo de saída cujo volume é a metade 
do nível original.

A Listagem~\ref{lst:video-filter} mostra o código de um programa que aplica
diferentes filtros em um vídeo de acordo com um argumento passado como
parâmetro (\C{argv[1]}). 

\lstinputlisting[
style=display,
mathescape=no,
caption={Adicionando diversos efeitos em um vídeo.},
label={lst:video-filter},
]{src/video_filter_v1.c}

O programa utiliza o elemento \C{uridecodebin} para
abrir o arquivo de vídeo ``bunny.ogg'' e decodificá-lo. A 
Figura~\ref{fig:pipe-filtro} ilustra esquematicamente o ``pipeline'' da
Listagem~\ref{lst:video-filter}. Dependendo do valor do argumento \C{argv[1]},
o elemento ``filtro'' da figura é instanciado como um elemento diferente.
Nesse pipeline, há um elemento \C{videoconvert} antes do filtro e outro após.
Como cada filtro manipula um determinado formato de \en{pixel}, a inclusão
destes elementos torna-se necessária para a realização das conversões entre
os formatos de \en{pixel} dos elementos anteriores e posteriores ao filtro.
Dependendo do filtro instanciado, ocorrerá um erro na conexão do pipeline
no caso de ausência desses conversores.

\begin{figure}[H]
  \centering
  \begin{tikzpicture}
    \node (dec) [element] {uridecodebin};
    \coordinate [above=.5\hdim of dec] (A);
    \coordinate [below=.5\hdim of dec] (B);
    \coordinate [right of=dec] (C);

    \node (videoconvert) [element] at (C|-A) {videoconvert};    
    \node (filter) [element, right of=videoconvert] {filtro};
    \node (videoconvert2) [element, right of=filter] {videoconvert};
    \node (xvimagesink) [element, right of=videoconvert2] {xvimagesink};
    \node (audioconvert) [element] at (C|-B) {audioconvert};
    \node (alsasink) [element, right of=audioconvert] {alsasink};
    \coordinate (X) at ($(dec.east)+(0,.25\hdim)$);
    \coordinate (Y) at ($(dec.east)+(0,-.25\hdim)$);
    \draw [->, arrow] (X) -- node [arrowlabel, above] {V} ++(\odim/3,0)
                          -- ($(videoconvert.west)-(\odim/3,0)$)
                          -- (videoconvert.west);
    \draw [->, arrow] (Y) -- node [arrowlabel, below] {A} ++(\odim/3,0)
                          -- ($(audioconvert.west)-(\odim/3,0)$)
                          -- (audioconvert.west);
    \draw [->, arrow] (videoconvert) -- (filter);
    \draw [->, arrow] (filter) -- (videoconvert2);
    \draw [->, arrow] (videoconvert2) -- (xvimagesink);
    \draw [->, arrow] (audioconvert) -- (alsasink);
  \end{tikzpicture}
  \caption{Um \en{pipeline} GStreamer que adiciona diferentes filtros 
    em um vídeo.}
  \label{fig:pipe-filtro}
\end{figure}
\vskip-\baselineskip
