\documentclass{SBCbookchapter}
\usepackage[utf8]{inputenc}
\usepackage[T1]{fontenc}
\usepackage[english,brazilian]{babel}
\usepackage{graphicx}

\title{Desenvolvimento de aplicações multimídia usando o framework GStreamer}
\author{XXX, YYY, ZZZ}

\begin{document}
\maketitle
\begin{abstract}
\begin{otherlanguage}{english}
\end{otherlanguage}
\end{abstract}

\begin{resumo}
\end{resumo}

\section{Público-alvo}
Programadores, alunos de graduação e pós-graduação interessados em
desenvolver aplicações multimídia usando o framekwork GStreamer~\cite{?}.  
Os alunos devem possuir conhecimento básico da linguagem~C~\cite{?} e estarem
familiarizados com algum ambiente de desenvolvimento para essa linguagem.
Como o minicurso será ministrado em um ambiente Linux, é desejável
que o público tenha conhecimento básico de alguma distribuição.  


\section{Sumário}
\begin{enumerate}
\item\emph{Introdução ao GStreamer}.  Apresentamos o \emph{framework}
  multimídia GStreamer: histórico, arquitetura básica, licença, quem usa,
  dependências, formatos e plataformas suportadas.  E discutimos brevemente
  o seu modelo de programação, viz., \emph{dataflow} (ou \emph{pipeline})
  multimídia~\cite{?}.

\item\emph{Hello-world: Tocando um vídeo}.  Apresentamos uma aplicação
  GStreamer básica que toca um vídeo e termina.  Esse exemplo utiliza a API
  de alto-nível~\texttt{GstPlayer}---atualmente a forma mais simples de se
  reproduzir um áudio ou vídeo no GStreamer.

\item\emph{Conceitos básicos: Destrinchando o exemplo anterior}.  Discutimos
  o que está por trás do código aparentemente simples do exemplo anterior.
  Começamos mostrando o \emph{pipeline} que a API~\texttt{GstPlayer} usa
  para reproduzir o vídeo e aproveitamos esse exemplo para introduzir cada
  um dos conceitos básicos do \emph{framework}:  \emph{element}, \emph{pad},
  \emph{caps}, \emph{clock}, \emph{buffer}, \emph{event}, \emph{message},
  \emph{bus}, \emph{bin} e \emph{pipeline}.  Após discutir os conceitos,
  voltamos à pratica e reimplementamos o exemplo do tópico anterior usando a
  API básica do GStreamer.  (Apesar da API~\texttt{GstPlayer} simplificar a
  programação de reprodutores de mídia, ela é inflexível e limita os
  possíveis usos do framework.  Dessa forma, é importante que os alunos do
  minicurso conheçam a API básica do GStreamer e seu modo de operação.)
  Ainda neste tópico, discutimos as ferramentas~\texttt{gst-inspect}, que
  consulta os elementos instalados no sistema, e ~\texttt{gst-launch}, que
  constrói um \emph{pipeline} na linha de comando.

\item\emph{Entrada e saída}.  Adicionamos suporte à entrada do usuário
  (teclado e mouse) ao exemplo anterior, e discutimos como os
  elementos~\texttt{appsrc} e~\texttt{appsink} podem ser utilizados para
  injetar ou obter dados do \emph{pipeline}.  Concluímos o tópico mostrando
  como renderizar os quadros produzidos no exemplo numa
  janela~GTK+~\cite{?}.

\item\emph{Filtros}.  Apresentamos os principais filtros de aúdio e vídeo
  disponíveis no GStreamer e discutimos como integrá-los ao exemplo.  Nesse
  ponto, apresentamos o resultado da combinação do exemplo original com
  diferentes filtros, e discutimos como os parâmetros dos filtros podem ser
  alterados dinamicamente em tempo de execução.

\item\emph{Pause, seek, fast-forward, rewind e step}.  Discutimos o
  funcionamento e implementamos as operações \emph{pause}, \emph{seek},
  \emph{fast-forward}, \emph{rewind} e \emph{frame stepping} no exemplo.
  Analisamos os problemas envolvidos em cada uma dessas operações destacando
  as situações em que elas podem falhar.

\item\emph{Plugins}.  Apresentamos brevemente a arquitetura de
  \emph{plugins} do GStreamer e discutimos a implementação de um filtro de
  vídeo simples.  O filtro utiliza a biblioteca de desenho vetorial
  Cairo~\cite{?} para desenhar sobre quadros de vídeo.  Além do código,
  mostramos como gerar um \emph{plugin} com esse filtro, instalá-lo, e
  utilizá-lo em conjunto com o exemplo original.

\item\emph{Conclusão}.  Terminamos o minicurso discutindo brevidente alguns
  tópicos avançados que não foram abordados---e.g., alterações na topologia
  do \emph{pipeline} em tempo de execução, \emph{mixers}, sincronização de
  pipelines, captura de áudio e vídeo, transmissão na rede, \emph{bindings}
  em outras linguagens, etc.---e listando nossas referências.
\end{enumerate}


\section{Justificativa do interesse do curso para o público alvo}

GStreamer~\cite{?} é um \emph{framework} multimídia escrito em C. Possui
uma arquitetura modularizada facilmente extensível por plugins, 
tornando-o bastante flexível. Atualmente é um dos principais \emph{frameworks} 
de código aberto para o desenvolvimento de aplicações multimídia, sendo
utilizado como backend em diversos softwares como Totem~\cite{?}, AmaroK~\cite{?},
Rhythmbox~\cite{?}, GNOME Videos~\cite{?}, entre outros.

O GStreamer usa a abstração de pipeline por onde os dados fluem 
a partir de uma fonte de dados, são processados por diversos componentes 
intermediários até finalmente serem renderizados, salvos em disco ou
enviados pela rede. Desenvolver aplicações usando o GStreamer pode ser
descrito basicamente como a especificação da composição dos componentes
de um pipeline que processam o conteúdo multimídia. O \emph{framework} 
suporta a sincronização de múltiplos objetos de mídia em uma apresentação
multimídia unificada.  

O \emph{framework} pode ser dividido em três partes: \emph{core}, \emph{plugins}
e ferramentas. O \emph{core} implementa a arquitetura base que suporta o
fluxo de dados em um pipeline composto por diversos \emph{plugins}. 
\emph{Plugins} são interligados entre si e processam o conteúdo multimídia
(i.e., de/codificação, de/multiplexação, filtros, renderização, etc).
Finalmente, diversas ferramentas estão disponíveis para facilitar
o uso do \emph{framework}. Exemplos de ferramentas são o 
\emph{gst-inspector} (lista os plugins instalados) e o \emph{gst-launch} 
(utilizado para construir pipelines simples via linha de comando).

Atualmente há pouca documentação atualizada disponível e a maior parte
está em inglês. Este minicurso visa disseminar o uso do GStreamer na comunidade 
brasileira de multimídia, apresentando os conceitos básicos do 
\emph{framework}. Espera-se ao final do minicurso que o público
consiga criar aplicações multimídia compostas por vídeos e áudios,
adicionando efeitos sobre os mesmos. Espera-se também que a audiência
consiga estender o \emph{framework} criando \emph{plugins} simples.

%Além disso, a maior parte dessa documentação está em Inglês.
%[Colocar mais coisa aqui, e.g., disponibilidade, flexibilidade, controle
%  fino, base para aplicações mais complexas, etc.]


\section{Currículo dos autores}

\noindent\emph{Guilherme F.~Lima}

\noindent\emph{Rodrigo Costa}

\noindent\emph{Roberto Gerson}


\section{Apresentador do minicurso}


\section{Recursos necessários}
\begin{itemize}
  \item Computadores com ambiente Linux instalado para os alunos
  \item Projeto multimídia 
\end{itemize}


\section*{Referências}
\end{document}
