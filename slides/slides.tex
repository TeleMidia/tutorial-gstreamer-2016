\documentclass{beamer}

\mode<presentation>{
  \usetheme{puc}
  \usecolortheme{beaver}
  \setbeamertemplate{caption}[numbered]
}

\usepackage[utf8]{inputenc}
\usepackage[brazilian]{babel}
\uselanguage{Brazilian}
\languagepath{Brazilian}
\usepackage{hyperref}
\usepackage{mathtools}
\usepackage{listings}
\usepackage{color}
\usepackage[normalem]{ulem}
\usepackage{textcomp}
\usepackage{tikz}
\usetikzlibrary{arrows,calc,positioning,shadows,shapes,trees}
\newdimen\hdim                  % element height
\newdimen\wdim                  % element width
\newdimen\odim                  % offset distance
\hdim=2.25em
\wdim=2.36\hdim
\odim=\hdim
\tikzstyle{element}=[
  draw=black!60,
  font={\scriptsize\sffamily},
  inner sep=0pt,
  minimum height=\hdim,
  minimum width=\wdim,
  outer sep=0pt,
  rectangle,
  rounded corners=\wdim/10,
  text centered,
  thick,
  top color=white,
  bottom color=black!20,
]
\tikzstyle{arrow}=[
  color=black!60,
  draw=black!60,
  thick,
]
\tikzstyle{arrowlabel}=[
  color=black!60,
  font={\tiny\sffamily\bfseries},
]
\tikzstyle{bin}=[
  dashed,
  draw=black!60,
  font={\tiny\sffamily\bfseries},
  rounded corners=\wdim/12,
]
\tikzstyle{binlabel}=[
  anchor=west,
  color=black!60,
  inner sep=0pt,
  outer sep=0pt,
  pos=0,
  xshift=.25em,
  yshift=-.45em,
]
\tikzstyle{state}=[
  circle,
  draw,
  font={\scriptsize\sffamily},
  inner sep=0pt,
  minimum height=1.5\hdim,
  minimum width=1.5\hdim,
  text width=\hdim,
  outer sep=0pt,
  text centered,
]
\tikzstyle{keystroke}=[
  draw,
  drop shadow={
    shadow xshift=0.25ex,
    shadow yshift=-0.25ex,
    fill=black,
    opacity=0.75,
  },
  fill=white,
  font=\scriptsize\sffamily,
  inner sep=2.5pt,
  line width=0.5pt,
  minimum width={1.7em},
  rectangle,
  rounded corners=2pt,
]
\tikzset{
  node distance=\wdim+\odim,
  >=stealth,
  shorten >=.5pt,
}

\newcommand*\keystroke[1]{%
  \tikz[baseline=(key.base)]\node[keystroke](key) {#1\strut};
}

\def\en#1{\foreignlanguage{english}{\emph{#1}}}

\lstset{
  basicstyle=\footnotesize\ttfamily,
  keywordstyle=\bfseries,
  breaklines=false,
  keepspaces=true,
  abovecaptionskip=0pt,
  aboveskip=\smallskipamount,
  belowcaptionskip=0pt,
  belowskip=\smallskipamount,
  captionpos=b,
  numbers=left,
  numberstyle=\tiny,
  numbersep=5pt,
  upquote,
  escapeinside={<@}{@>},
}

\lstdefinestyle{display}{
  aboveskip=\abovedisplayskip,
  basicstyle=\tiny\ttfamily,
  belowskip=0pt,
  captionpos=b,
  numbers=left,
  numberstyle={\tiny\sffamily},
}

\lstdefinestyle{command}{
  basicstyle=\small\ttfamily,
  aboveskip=\abovedisplayskip,
  belowskip=.5\belowdisplayskip,
  numbers=none,
}

\hypersetup{
  colorlinks=true,
  urlcolor=blue,
}


\title[GStreamer]{Programando Aplicações Multimídia no GStreamer}
\author{\
  Guilherme Lima                  \quad
  \textcolor{red}{Rodrigo Costa}  \quad
  Roberto Gerson
}
\institute[TeleMídia Lab.]
{
  Pontifícia Universidade Católica -- PUC-Rio\\ 
  TeleMídia Lab., Dep. de Informática.

  \medskip
  \textit{\{gflima,rodrigocosta,robertogerson\}@telemidia.puc-rio.br}
}
\date{\today}

\begin{document}
\frame{\titlepage}

\begin{frame}[c]{Quem somos}
  \begin{block}{Guilherme Lima}
    \begin{itemize}
      \item Doutor em Informática pela PUC-Rio (2015)
      \item Interesses de Pesquisa: linguagens de programação e modelos
        para sincronismo multimídia. 
    \end{itemize}
  \end{block}
  \begin{block}{Rodrigo Costa}
    \begin{itemize}
      \item Doutorando em Informática na PUC-Rio
      \item Interesses de Pesquisa: sistemas multimídia distribuídos
      e modelos síncronos para autoria multimídia. 
    \end{itemize}
  \end{block}
  \begin{block}{Roberto Gerson}
    \begin{itemize}
      \item Doutor em Informática pela PUC-Rio (2015)
      \item Interesses de Pesquisa: representação e autoria de cenas
        multimídia interativas e representação e renderização de vídeos 3D.
    \end{itemize}
  \end{block}
\end{frame}

\begin{frame}[c]{Acessando o Repositório do Minicurso}
  \url{https://github.com/TeleMidia/minicurso-webmedia16}

  \begin{itemize}
    \item Pasta \en{src} 
      \begin{itemize}
        \item códigos fonte utilizados neste minicurso
      \end{itemize}
    \item Pasta \en{slides} 
      \begin{itemize}
        \item slides utilizados neste minicurso
      \end{itemize}
  \end{itemize}
\end{frame}

\begin{frame}{\contentsname}
  \tableofcontents
\end{frame}

\section{Introdução}
\label{sub:overview}

\begin{frame}[c]{GStreamer}
  \begin{itemize}
    \item Um dos principais \en{frameworks} de código aberto 
      para processamento de dados multimídia
      \begin{itemize}
        \item Projeto com mais de 15 anos
      \end{itemize}
    \item Projetado facilitar o desenvolvimento de aplicações que apresentam
      ou processam conteúdo audiovisual
    \item Softwares que usam o GStreamer:
      \begin{itemize}
        \item \href{https://amarok.kde.org/}{amaroK}
        \item \href{http://banshee.fm/}{Banshee}
        \item \href{http://eina.sourceforge.net/}{Eina}
        \item \href{https://wiki.gnome.org/Apps/Empathy}{Empathy}
        \item \href{http://www.rhythmbox.org/}{Rhythmbox}
        \item \href{https://wiki.gnome.org/Apps/Videos}{Totem}
      \end{itemize}
  \end{itemize}
\end{frame}

\begin{frame}[c]{GStreamer}
  \begin{itemize}
    \item Multiplataforma 
    \item Robusto
    \item Flexível
    \item Extensível por Plugins
    \item APIs de alto e baixo nível
    \item Baseado no modelo de computação \en{dataflow} 
      \begin{itemize}
        \item \en{``Pipeline"} na terminologia do GStreamer
      \end{itemize}
    \item Desenvolvido em C
      \begin{itemize}
        \item Possui \en{bindings} para outras linguagens
      \end{itemize}
  \end{itemize}

  \begin{block}{O que o GStreamer NÃO é}
    \begin{itemize}
      \item Uma implementação de CODEC
      \item Uma aplicação \en{standalone}
    \end{itemize}
  \end{block}
\end{frame}

\begin{frame}[c]{GStreamer}
  \begin{figure}
    \centering
    \includegraphics[scale=0.5]{figs/gstreamer-overview}
  \end{figure}
\end{frame}

\begin{frame}[c]{Dataflow}
  \begin{itemize}
    \item Dados são processados enquanto ``fluem" através de uma rede
    \item Estrutura de grafo dirigido
      \begin{itemize}
        \item Nós representam elementos de processamento (atores)
        \item Arestas representam conexões unidirecionais por onde fluem 
          os dados
      \end{itemize}
    \item Atores recebem dados em portas de entrada e emitem dados através
      de portas de saída
    \item Um \en{pipeline} é um \en{dataflow} em que os dados fluem
      na mesma ordem em que foram produzidos
  \end{itemize}

  \begin{figure}[H]
    \centering
    \begin{tikzpicture}
      \node (filesrc) [element] {filesrc};
      \coordinate [above=.5\hdim of filesrc] (A);
      \coordinate [below=.5\hdim of filesrc] (B);
      %%
      \node (oggdemux) [element, right of=filesrc] {oggdemux};
      \coordinate [right of=oggdemux] (C);
      %%
      \node (vorbisdec) [element] at (C|-A) {vorbisdec};
      \node (alsasink) [element, right of=vorbisdec] {alsasink};
      \node (theoradec) [element] at (C|-B) {theoradec};
      \node (xvimagesink) [element, right of=theoradec] {xvimagesink};
      %%
      \draw [->, arrow] (filesrc) -- (oggdemux);
      \coordinate (X) at ($(oggdemux.east)+(0,.25\hdim)$);
      \coordinate (Y) at ($(oggdemux.east)+(0,-.25\hdim)$);
      \draw [->, arrow] (X) -- node [arrowlabel, above] {A} ++(\odim/3,0)
      -- ($(vorbisdec.west)-(\odim/3,0)$)
      -- (vorbisdec.west);
      \draw [->, arrow] (Y) -- node [arrowlabel, below] {V} ++(\odim/3,0)
      -- ($(theoradec.west)-(\odim/3,0)$)
      -- (theoradec.west);
      \draw [->, arrow] (vorbisdec) -- (alsasink);
      \draw [->, arrow] (theoradec) -- (xvimagesink);
    \end{tikzpicture}
  \end{figure}
\end{frame}

\begin{frame}[c]{Pipelines no GStreamer}
  \begin{itemize}
    \item Em um \en{pipeline} GStreamer os dados que fluem são tipicamente
      amostras de áudio e vídeo e dados de controle
    \item Nós do grafo (atores) são chamados elementos
    \item Portas por onde entram e saem dados dos elementos são chamados de
      \en{pads}
      \begin{itemize}
        \item \en{Sink pad} -- portas de entrada 
        \item \en{Source pad} -- portas de saída 
      \end{itemize}
    \item Tipos de elementos
      \begin{itemize}
        \item \en{Source} (produtores) -- possuem apenas \en{source pads}
        \item Processadores -- possuem \en{source} e \en{sink pads}
        \item \en{Sink} (consumidores) -- possuem apenas \en{sink pads}
      \end{itemize}
  \end{itemize}
\end{frame}

\begin{frame}[c]{Pipelines no GStreamer}
  Um \en{pipeline} GStreamer simplificado que reproduz um arquivo de 
  vídeo Ogg.

  \begin{itemize}
    \item<2> \en{Source}
      \begin{itemize}
        \item filesrc
      \end{itemize}

    \item<3> Processadores
      \begin{itemize}
        \item oggdemux 
        \item vorbisdec
        \item theoradec 
      \end{itemize}

    \item<4> \en{Sinks}
      \begin{itemize}
        \item alsasink 
        \item xvimagesink 
      \end{itemize}
  \end{itemize}
  \newcommand*\overlaytwo{}
  \newcommand*\overlaythree{}
  \newcommand*\overlayfour{}
  \only<2>{\renewcommand*\overlaytwo{red}}
  \only<3>{\renewcommand*\overlaythree{red}}
  \only<4>{\renewcommand*\overlayfour{red}}

  \begin{figure}[h]
    \centering
    \begin{tikzpicture}
      \node (filesrc) [element, \overlaytwo] {filesrc};
      \coordinate [above=.5\hdim of filesrc] (A);
      \coordinate [below=.5\hdim of filesrc] (B);
      %%
      \node (oggdemux) [element, right of=filesrc, \overlaythree] {oggdemux};
      \coordinate [right of=oggdemux] (C);
      %%
      \node (vorbisdec) [element, \overlaythree] at (C|-A) {vorbisdec};
      \node (alsasink) [element, right of=vorbisdec, \overlayfour] {alsasink};
      \node (theoradec) [element, \overlaythree] at (C|-B) {theoradec};
      \node (xvimagesink) [element, right of=theoradec, \overlayfour] {xvimagesink};
      %%
      \draw [->, arrow] (filesrc) -- (oggdemux);
      \coordinate (X) at ($(oggdemux.east)+(0,.25\hdim)$);
      \coordinate (Y) at ($(oggdemux.east)+(0,-.25\hdim)$);
      \draw [->, arrow] (X) -- node [arrowlabel, above] {A} ++(\odim/3,0)
      -- ($(vorbisdec.west)-(\odim/3,0)$)
      -- (vorbisdec.west);
      \draw [->, arrow] (Y) -- node [arrowlabel, below] {V} ++(\odim/3,0)
      -- ($(theoradec.west)-(\odim/3,0)$)
      -- (theoradec.west);
      \draw [->, arrow] (vorbisdec) -- (alsasink);
      \draw [->, arrow] (theoradec) -- (xvimagesink);
    \end{tikzpicture}
  \end{figure}
\end{frame}

\begin{frame}[c]{Pipelines no GStreamer}
  \begin{block}{filesrc}
    Lê um arquivo do disco (vamos assumir um arquivo Ogg) e escreve o 
    fluxo de bytes resultante na sua \en{source pad}
  \end{block}

  \begin{figure}
    \centering
    \begin{tikzpicture}
      \node (filesrc) [element, red] {filesrc};
      \coordinate [above=.5\hdim of filesrc] (A);
      \coordinate [below=.5\hdim of filesrc] (B);
      %%
      \node (oggdemux) [element, right of=filesrc] {oggdemux};
      \coordinate [right of=oggdemux] (C);
      %%
      \node (vorbisdec) [element] at (C|-A) {vorbisdec};
      \node (alsasink) [element, right of=vorbisdec] {alsasink};
      \node (theoradec) [element] at (C|-B) {theoradec};
      \node (xvimagesink) [element, right of=theoradec] {xvimagesink};
      %%
      \draw [->, arrow] (filesrc) -- (oggdemux);
      \coordinate (X) at ($(oggdemux.east)+(0,.25\hdim)$);
      \coordinate (Y) at ($(oggdemux.east)+(0,-.25\hdim)$);
      \draw [->, arrow] (X) -- node [arrowlabel, above] {A} ++(\odim/3,0)
      -- ($(vorbisdec.west)-(\odim/3,0)$)
      -- (vorbisdec.west);
      \draw [->, arrow] (Y) -- node [arrowlabel, below] {V} ++(\odim/3,0)
      -- ($(theoradec.west)-(\odim/3,0)$)
      -- (theoradec.west);
      \draw [->, arrow] (vorbisdec) -- (alsasink);
      \draw [->, arrow] (theoradec) -- (xvimagesink);
    \end{tikzpicture}
  \end{figure}
\end{frame}

\begin{frame}[c]{Pipelines no GStreamer}
  \begin{block}{oggdemux}
    Lê da sua \en{sink pad} um fluxo de bytes codificado no formato Ogg, 
    demultiplexa-o e escreve os fluxos Vorbis (áudio) e Theora (vídeo) 
    resultantes nas \en{source pads} correspondentes
  \end{block}

  \begin{figure}
    \centering
    \begin{tikzpicture}
      \node (filesrc) [element] {filesrc};
      \coordinate [above=.5\hdim of filesrc] (A);
      \coordinate [below=.5\hdim of filesrc] (B);
      %%
      \node (oggdemux) [element, right of=filesrc, red] {oggdemux};
      \coordinate [right of=oggdemux] (C);
      %%
      \node (vorbisdec) [element] at (C|-A) {vorbisdec};
      \node (alsasink) [element, right of=vorbisdec] {alsasink};
      \node (theoradec) [element] at (C|-B) {theoradec};
      \node (xvimagesink) [element, right of=theoradec] {xvimagesink};
      %%
      \draw [->, arrow] (filesrc) -- (oggdemux);
      \coordinate (X) at ($(oggdemux.east)+(0,.25\hdim)$);
      \coordinate (Y) at ($(oggdemux.east)+(0,-.25\hdim)$);
      \draw [->, arrow] (X) -- node [arrowlabel, above] {A} ++(\odim/3,0)
      -- ($(vorbisdec.west)-(\odim/3,0)$)
      -- (vorbisdec.west);
      \draw [->, arrow] (Y) -- node [arrowlabel, below] {V} ++(\odim/3,0)
      -- ($(theoradec.west)-(\odim/3,0)$)
      -- (theoradec.west);
      \draw [->, arrow] (vorbisdec) -- (alsasink);
      \draw [->, arrow] (theoradec) -- (xvimagesink);
    \end{tikzpicture}
  \end{figure}
\end{frame}

\begin{frame}[c]{Pipelines no GStreamer}
  \begin{block}{vorbisdec}
    Lê da sua \en{sink pad} um fluxo de bytes
    codificado no formato Vorbis, decodifica-o e escreve o fluxo de áudio~PCM
    resultante (áudio \en{raw} descomprimido) na sua \en{source pad}
  \end{block}

  \begin{figure}
    \centering
    \begin{tikzpicture}
      \node (filesrc) [element] {filesrc};
      \coordinate [above=.5\hdim of filesrc] (A);
      \coordinate [below=.5\hdim of filesrc] (B);
      %%
      \node (oggdemux) [element, right of=filesrc] {oggdemux};
      \coordinate [right of=oggdemux] (C);
      %%
      \node (vorbisdec) [element, red] at (C|-A) {vorbisdec};
      \node (alsasink) [element, right of=vorbisdec] {alsasink};
      \node (theoradec) [element] at (C|-B) {theoradec};
      \node (xvimagesink) [element, right of=theoradec] {xvimagesink};
      %%
      \draw [->, arrow] (filesrc) -- (oggdemux);
      \coordinate (X) at ($(oggdemux.east)+(0,.25\hdim)$);
      \coordinate (Y) at ($(oggdemux.east)+(0,-.25\hdim)$);
      \draw [->, arrow] (X) -- node [arrowlabel, above] {A} ++(\odim/3,0)
      -- ($(vorbisdec.west)-(\odim/3,0)$)
      -- (vorbisdec.west);
      \draw [->, arrow] (Y) -- node [arrowlabel, below] {V} ++(\odim/3,0)
      -- ($(theoradec.west)-(\odim/3,0)$)
      -- (theoradec.west);
      \draw [->, arrow] (vorbisdec) -- (alsasink);
      \draw [->, arrow] (theoradec) -- (xvimagesink);
    \end{tikzpicture}
  \end{figure}
\end{frame}

\begin{frame}[c]{Pipelines no GStreamer}
  \begin{block}{theoradec}
    Lê da sua \en{sink pad} um fluxo de bytes codificado no formato 
    Theora, decodifica-o e escreve o fluxo de vídeo \en{raw} descomprimido 
    resultante na sua \en{source pad}
  \end{block}

  \begin{figure}
    \centering
    \begin{tikzpicture}
      \node (filesrc) [element] {filesrc};
      \coordinate [above=.5\hdim of filesrc] (A);
      \coordinate [below=.5\hdim of filesrc] (B);
      %%
      \node (oggdemux) [element, right of=filesrc] {oggdemux};
      \coordinate [right of=oggdemux] (C);
      %%
      \node (vorbisdec) [element] at (C|-A) {vorbisdec};
      \node (alsasink) [element, right of=vorbisdec] {alsasink};
      \node (theoradec) [element, red] at (C|-B) {theoradec};
      \node (xvimagesink) [element, right of=theoradec] {xvimagesink};
      %%
      \draw [->, arrow] (filesrc) -- (oggdemux);
      \coordinate (X) at ($(oggdemux.east)+(0,.25\hdim)$);
      \coordinate (Y) at ($(oggdemux.east)+(0,-.25\hdim)$);
      \draw [->, arrow] (X) -- node [arrowlabel, above] {A} ++(\odim/3,0)
      -- ($(vorbisdec.west)-(\odim/3,0)$)
      -- (vorbisdec.west);
      \draw [->, arrow] (Y) -- node [arrowlabel, below] {V} ++(\odim/3,0)
      -- ($(theoradec.west)-(\odim/3,0)$)
      -- (theoradec.west);
      \draw [->, arrow] (vorbisdec) -- (alsasink);
      \draw [->, arrow] (theoradec) -- (xvimagesink);
    \end{tikzpicture}
  \end{figure}
\end{frame}

\begin{frame}[c]{Pipelines no GStreamer}
  \begin{block}{alsasink}
    Lê um fluxo de áudio descomprimido da sua \en{sink pad} e utiliza 
    a biblioteca ALSA para reproduzir as amostras do fluxo 
    nos alto-falantes
  \end{block}

  \begin{figure}
    \centering
    \begin{tikzpicture}
      \node (filesrc) [element] {filesrc};
      \coordinate [above=.5\hdim of filesrc] (A);
      \coordinate [below=.5\hdim of filesrc] (B);
      %%
      \node (oggdemux) [element, right of=filesrc] {oggdemux};
      \coordinate [right of=oggdemux] (C);
      %%
      \node (vorbisdec) [element] at (C|-A) {vorbisdec};
      \node (alsasink) [element, right of=vorbisdec, red] {alsasink};
      \node (theoradec) [element] at (C|-B) {theoradec};
      \node (xvimagesink) [element, right of=theoradec] {xvimagesink};
      %%
      \draw [->, arrow] (filesrc) -- (oggdemux);
      \coordinate (X) at ($(oggdemux.east)+(0,.25\hdim)$);
      \coordinate (Y) at ($(oggdemux.east)+(0,-.25\hdim)$);
      \draw [->, arrow] (X) -- node [arrowlabel, above] {A} ++(\odim/3,0)
      -- ($(vorbisdec.west)-(\odim/3,0)$)
      -- (vorbisdec.west);
      \draw [->, arrow] (Y) -- node [arrowlabel, below] {V} ++(\odim/3,0)
      -- ($(theoradec.west)-(\odim/3,0)$)
      -- (theoradec.west);
      \draw [->, arrow] (vorbisdec) -- (alsasink);
      \draw [->, arrow] (theoradec) -- (xvimagesink);
    \end{tikzpicture}
  \end{figure}
\end{frame}

\begin{frame}[c]{Pipelines no GStreamer}
  \begin{block}{xvimagesink}
    Lê um fluxo de vídeo descomprimido da sua \en{sink pad} e utiliza 
    a biblioteca X11 para reproduzir os quadros do fluxo na tela
  \end{block}

  \begin{figure}
    \centering
    \begin{tikzpicture}
      \node (filesrc) [element] {filesrc};
      \coordinate [above=.5\hdim of filesrc] (A);
      \coordinate [below=.5\hdim of filesrc] (B);
      %%
      \node (oggdemux) [element, right of=filesrc] {oggdemux};
      \coordinate [right of=oggdemux] (C);
      %%
      \node (vorbisdec) [element] at (C|-A) {vorbisdec};
      \node (alsasink) [element, right of=vorbisdec] {alsasink};
      \node (theoradec) [element] at (C|-B) {theoradec};
      \node (xvimagesink) [element, right of=theoradec, red] {xvimagesink};
      %%
      \draw [->, arrow] (filesrc) -- (oggdemux);
      \coordinate (X) at ($(oggdemux.east)+(0,.25\hdim)$);
      \coordinate (Y) at ($(oggdemux.east)+(0,-.25\hdim)$);
      \draw [->, arrow] (X) -- node [arrowlabel, above] {A} ++(\odim/3,0)
      -- ($(vorbisdec.west)-(\odim/3,0)$)
      -- (vorbisdec.west);
      \draw [->, arrow] (Y) -- node [arrowlabel, below] {V} ++(\odim/3,0)
      -- ($(theoradec.west)-(\odim/3,0)$)
      -- (theoradec.west);
      \draw [->, arrow] (vorbisdec) -- (alsasink);
      \draw [->, arrow] (theoradec) -- (xvimagesink);
    \end{tikzpicture}
  \end{figure}
\end{frame}

\begin{frame}[c]{Sincronização}
  \begin{itemize}
    \item \en{Pipelines} possuem um relógio para controlar a sincronização 
      dos fluxos
    \item Cada amostra de áudio e vídeo possuem um tempo de apresentação
      (PTS -- \en{presentation timestamp}) e uma duração
    \item Elementos \en{sink} controlam a taxa de reprodução de cada
      fluxo
      \begin{itemize}
        \item Amostras recebidas antes do seu tempo de apresentação são 
          armazenadas em uma fila interna para serem exibidas no 
          tempo adequado
        \item Amostras recebidas após o seu tempo de apresentação são 
          descartadas 
      \end{itemize}
    \item Todos os outros elementos do \en{pipeline} operam livremente
      \begin{itemize}
        \item Consumem e produzem dados em taxas arbitrárias
      \end{itemize}
  \end{itemize}
\end{frame}

\begin{frame}[c]{Pads}
  \begin{itemize}
    \item \en{Pads} são pontos de conexão entre elementos
      \begin{itemize}
        \item source $\rightarrow$ sink \textcolor{green}{\checkmark}
        \item sink $\rightarrow$ source $\color{red}{\times}$
      \end{itemize}
    \item Dois tipos de dados trafegam entre \en{pads}
      \begin{itemize}
        \item Dados (\en{buffers})
          \begin{itemize}
            \item Amostras de áudio/vídeo
            \item Fluem exclusivamente na direção das conexões 
          \end{itemize}
        \item Eventos (\en{events})
          \begin{itemize}
            \item Informações de controle 
            \item Podem fluir em ambos os sentidos das conexões 
            \item Ex: QoS, seek, flush, \ldots
          \end{itemize}
      \end{itemize}
    \item Dados e eventos podem percorrer as conexões em paralelo 
  \end{itemize}

  \begin{figure}[H]
    \centering
    \begin{tikzpicture}
      \node [cylinder, thick, draw=black!60, cylinder uses custom fill,
      cylinder end fill=black!30, cylinder body fill=black!20,
      minimum height=\wdim, minimum width=\hdim] (c) {};
      \coordinate (x) at ($(c.before top|-c.east)+(0,.15\hdim)$);
      \coordinate (y) at ($(c.before top|-c.east)-(0,.15\hdim)$);
      \coordinate (x0) at ($(c.west|-x)+(.25pt,0)$);
      \coordinate (y0) at ($(c.west|-y)+(.25pt,0)$);
      \draw [->,arrow] (x)
      -- node [arrowlabel, above, pos=.4] {B} ++(\odim,0);
      \draw [arrow] (x0) -- ++(-\odim,0);
      \draw [->,arrow] (y)
      -- node [arrowlabel, below, pos=.4] {E} ++(\odim,0);
      \draw [->,arrow] (y0) -- ++(-\odim,0);
    \end{tikzpicture}
  \end{figure}
  \vskip-\baselineskip
  Estrutura conceitual de uma conexão entre \en{pads} no
  GStreamer.
\end{frame}

\section{Primeira Aplicação GStreamer: Olá mundo}
\begin{frame}[fragile]{Olá mundo -- playbin}
  Tocando um vídeo no GStreamer usando o elemento ``playbin''.
  \begin{itemize}
    \item Arquivo: src/hello.c
  \end{itemize}
  ~
  
  Compilando o código fonte hello.c:
  \begin{lstlisting}[style=command]
  $ cc hello.c -o hello `pkg-config --cflags --libs 
        glib-2.0 gstreamer-1.0`
  \end{lstlisting}

\end{frame}

\end{document}
